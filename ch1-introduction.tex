\chapter{General Introduction}
\label{chap:introduction}%Note this label will be used to refer to the chapter throughout. So if you change the order of chapters it still knows this one is this file, but can call it chapter 1 or 2 or whatever depending on the order. S oti's better than calling it chapter 1.

\begin{quoteshrink}
  ``Really grandiose sounding quotes from Darwin always make a thesis feel more professional''
  \hfill{Natalie Cooper, p.~15}
\end{quoteshrink}

\noindent
Predator-prey interactions are an important evolutionary driver and a central component of ecosystem structure. However, the context dependent nature of these interactions, as reflected by the diversity of species involved, means that understanding them at a fundamental level is required for purposeful predictions. In this thesis I explore the role of two fundamental components of predator-prey interactions: habitat dimensionality and body size. I investigate the fundamental role of body size and habitat dimensionality across three chapters that represent stand-alone publications consisting of: the role of body size on the ability of species to perceive the temporal dimension of their environment; the role of habitat dimensionality and other traits relating to predation pressure on life history evolution; and the role of habitat dimensionality on the evolution of venom toxicity in predatory snakes. Throughout my thesis I use a comparative approach to show that both habitat dimensionality and body size are key components that determine the mechanics of predator-prey interactions and hence ecological and evolutionary systems as a whole.


\section{Structure \& contents of this thesis}
In this thesis, I do some really cool stuff.
%
In \chapref{time}, I do some lab work.
%
In \chapref{longevity}, I train a velociraptor to ride a hoverboard.
%
%
Finally, in \chapref{conclusions}, I close with a discussion of the
limitations of the methods used in the thesis, and suggest some future
directions.

